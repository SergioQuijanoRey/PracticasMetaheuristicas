\documentclass[11pt]{article}

% Paquetes
%===============================================================================

% Paquete para incluir imagenes
\usepackage{graphicx}
\graphicspath{ {./images/} }

% Paquete para incluir trozos de codigo
\usepackage{listings, listings-rust}

% Para que el codigo acepte caracteres en utf8
\lstset{literate=
  {á}{{\'a}}1 {é}{{\'e}}1 {í}{{\'i}}1 {ó}{{\'o}}1 {ú}{{\'u}}1
  {Á}{{\'A}}1 {É}{{\'E}}1 {Í}{{\'I}}1 {Ó}{{\'O}}1 {Ú}{{\'U}}1
  {à}{{\`a}}1 {è}{{\`e}}1 {ì}{{\`i}}1 {ò}{{\`o}}1 {ù}{{\`u}}1
  {À}{{\`A}}1 {È}{{\'E}}1 {Ì}{{\`I}}1 {Ò}{{\`O}}1 {Ù}{{\`U}}1
  {ä}{{\"a}}1 {ë}{{\"e}}1 {ï}{{\"i}}1 {ö}{{\"o}}1 {ü}{{\"u}}1
  {Ä}{{\"A}}1 {Ë}{{\"E}}1 {Ï}{{\"I}}1 {Ö}{{\"O}}1 {Ü}{{\"U}}1
  {â}{{\^a}}1 {ê}{{\^e}}1 {î}{{\^i}}1 {ô}{{\^o}}1 {û}{{\^u}}1
  {Â}{{\^A}}1 {Ê}{{\^E}}1 {Î}{{\^I}}1 {Ô}{{\^O}}1 {Û}{{\^U}}1
  {ã}{{\~a}}1 {ẽ}{{\~e}}1 {ĩ}{{\~i}}1 {õ}{{\~o}}1 {ũ}{{\~u}}1
  {Ã}{{\~A}}1 {Ẽ}{{\~E}}1 {Ĩ}{{\~I}}1 {Õ}{{\~O}}1 {Ũ}{{\~U}}1
  {œ}{{\oe}}1 {Œ}{{\OE}}1 {æ}{{\ae}}1 {Æ}{{\AE}}1 {ß}{{\ss}}1
  {ű}{{\H{u}}}1 {Ű}{{\H{U}}}1 {ő}{{\H{o}}}1 {Ő}{{\H{O}}}1
  {ç}{{\c c}}1 {Ç}{{\c C}}1 {ø}{{\o}}1 {å}{{\r a}}1 {Å}{{\r A}}1
  {€}{{\euro}}1 {£}{{\pounds}}1 {«}{{\guillemotleft}}1
  {»}{{\guillemotright}}1 {ñ}{{\~n}}1 {Ñ}{{\~N}}1 {¿}{{?`}}1 {¡}{{!`}}1
}

% Para que los metadatos que escribe latex esten en español
\usepackage[spanish]{babel}

% Metadatos del documento
%===============================================================================
\title{
    {Práctica 1b}\\
    {Técnicas de Búsqueda Local y Algoritmos Greedy}\\
    {Problema de Agrupamiento con Restricciones}\\
}
\author{
    {Sergio Quijano Rey - 72103503k}\\
    {4º Doble Grado Ingeniería Informática y Matemáticas}\\
    {Grupo de prácticas 2 - Viernes 17.30h a 19.30h}\\
    {sergioquijano@correo.ugr.es}
}
\date{\today}

% Separacion entre parrafos
\setlength{\parskip}{1em}

% Contenidos del documento
%===============================================================================

\begin{document}

% Portada del documento
\maketitle
\pagebreak

% Indice de contenidos
\tableofcontents
\pagebreak

% Primer ejercicio
\section{Descripción del problema}

Vamos a trabajar el problema del agrupamiento con restricciones (\textbf{\emph{PAR}}). Consiste en una generalización del problema de agrupamiento clásico, al que añadimos restricciones sobre los datos.

El problema de agrupamiento clásico consiste en, dados unos datos de entrada sin etiquetar $X$ de tamaño $n$, agruparlos en $k$ grupos (o en inglés, \emph{clusters}) diferentes, formando una partición $C$ de $X$, de forma que se optimice alguna métrica. Normalmente, se busca minimizar la distancia \emph{intra\_cluster} (que más tarde se definirá).

La diferencia con el problema de agrupamiento clásico, por tanto, es la inclusión de restricciones. En nuestro caso concreto, trabajermos con restricciones entre pares de puntos, que además serán de dos tipos:

\begin{itemize}
    \item Restricción tipo \emph{Must Link}: los dos puntos afectados por esta restricción deberán pertenecer al mismo cluster
    \item Restricción tipo \emph{Cannot Link}: los dos puntos afectados por esta restricción no deben pertenecer al mismo cluster
\end{itemize}

Consideraremos de forma débil estas restricciones, es decir, podemos incumplir algunas restricciones. Pero supondrá que la solución será de peor calidad. Para especificar mejor esta noción, definimos la función de \emph{fitness} que buscamos minimizar:

\begin{displaymath}
    fitness(sol) := distancia_{intra-cluster}(sol) + \lambda * infeasibility(sol)
\end{displaymath}

donde $infeasibility$ es el número de restricciones que se incumplen. Esta función de $fitness$ nos permite pasar de intentar optimizar dos funciones objetivo a solo tener que optimizar un objetivo. El valor de $\lambda$ se especifica más adelante.

Como los datos no están etiquetados a priori, podríamos considerar este problema como un problema de aprendizaje no supervisado. Sin embargo, se puede considerar que las restricciones nos dan un tipo de etiquetado, por lo que es más correcto pensar que estamos ante una tarea de aprendizaje \emph{semi-supervisado}. La principal utilidad de resolver estos problemas es que normalmente estamos reduciendo la dimensionalidad de los datos a analizar, y de este modo, es más sencillo extraer conocimiento sobre dichos datos. También es útil para realizar

\pagebreak

% TODO -- es un máximo de 4 páginas y esto ocupa 5 páginas
% TODO -- describir en pseudo la funcion objetivo y los operadores comunes? ==> Apartado d
\section{Descripción de la aplicación de los algoritmos empleados}

\subsection{Representación del conjunto de datos}

Los datos vienen dados en una matrix de tamaño $n x d$ donde $n$ es el número de puntos y $d$ es la dimensión de cada uno de los puntos.

\subsubsection{Representación del conjunto de datos en código}

El conjunto de datos viene representado en \lstinline{problem_datatypes::DataPoints} que contiene un vector de otro tipo de dato: \lstinline{problem_datatypes::Point}. El tipo de dato \lstinline{Point} tiene un campo que es de tipo \lstinline{ndarray::Array1<f64>} que representa un vector (usamos una librería para trabajar con matrices y vectores). Por tanto, hemos pasado de trabajar con una matriz de datos a trabajar con un vector de puntos. Esto nos permite trabajar de forma más expresiva y sencilla con el problema. Por ejemplo, podemos calcular con métodos del \lstinline{struct} distancia entre dos puntos, centroide de un conjunto de puntos, \ldots

\subsection{Representación de las restricciones}

Las restricciones vienen dadas en un fichero de datos que representa una matriz que codifica las restricciones de la siguiente forma:

\begin{itemize}
    \item El elemento en la fila $i$-ésima y columna $j$-ésima representa las restricciones que hay entre el punto $i$ y el punto $j$
    \item Como la restricción que tenga el punto $i$ con el punto $j$ implica que el punto $j$ tiene la misma restricción con el punto $i$, es claro que dicha matriz debe ser simétrica
    \item Un valor $0$ significa que no hay restricciones. Un valor $1$ significa que hay una restricción tipo \emph{Must Link}. Un valor $-1$ implica una restricción \emph{Cannot Link}
    \item Además, la matriz tiene la diagonal de $1$s
\end{itemize}

\subsubsection{Representación de las restricciones en código}

El \lstinline{struct} \lstinline{problem_datatypes::Constraints} junto al enumerado \lstinline{problem_datatypes::ConstraintType} representan en el código las restricciones. El código es el siguiente:

\begin{lstlisting}[language=Rust, style=Boxed]
pub enum ConstraintType {
    MustLink,
    CannotLink,
}

pub struct Constraints{
    data: HashMap<(i32, i32), ConstraintType>,
}
\end{lstlisting}

Es claro que guardamos pares de enteros, que marcan los índices de los puntos, y la restricción entre el par de puntos representados, en un \lstinline{HashMap}. Esta elección viene motivada por:

\begin{itemize}
    \item Podemos acceder a las restricciones entre dos puntos en tiempo constante
    \item Podemos iterar sobre todas las restricciones, gracias a los métodos proporcionados por el lenguaje de programación, en un tiempo más que razonable. Así iteramos solo sobre una lista de $r$ restricciones, en vez de sobre una matriz cuadrada de dimensión $n^2$
    \item En cierto modo, estamos combinando los beneficios de tener acceso directo a elementos concretos y los beneficios de poder iterar sobre una lista (aunque iterar sobre un \lstinline{Hash} puede ser algo más lento que iterar sobre una lista o un \lstinline{array})
    \item Es fácil de implementar métodos para operar con restricciones con este tipo de dato
\end{itemize}

La implementación de los métodos que permiten manipular el \lstinline{struct} aseguran que:
\begin{itemize}
    \item No guardamos la restricción $(i, j)$ y junto a la $(j, i)$. Solo guardamos una de las dos restricciones, ahorrando memoria
    \item De hecho, el criterio es guardar como índices el par $(i, j)$ donde $i \leq j$
    \item Tampoco guardamos las restricciones $(i, i), MustLink$ pues son restricciones triviales
\end{itemize}


\subsection{Representación de la solución}

Una representación de la solución será un vector de tamaño $n$ con valores en $\{0, \ldots, k - 1\}$ donde $n$ es el número de puntos y $k$ es el número de clusters en los que dividimos los datos. Este vector representa la partición de los datos en los $k$ clusters. En la posición $i$-ésima del vector, guardamos el cluster al que pertecene el punto $i$-ésimo de nuestro conjunto de datos.

Las soluciones deben cumplir las siguientes restricciones:

\begin{itemize}
    \item No pueden quedar clusters vacíos. Es decir, clusters a los que no haya ningún punto asignado. Esto puede verse viendo que $\forall i \in \{0, \ldots, k-1\}$ $\exists pos \in \{0, \ldots, n-1\}$ tal que $solution[pos] = i$, es decir, el vector de soluciones tiene al menos una vez cada valor posible de los clusters
    \item Cada punto solo puede pertenecer a un único cluster. Por la forma vectorial en la que representamos la partición, esta restricción se verifica forzosamente, y por tanto no nos tenemos que preocupar de realizar comprobaciones
    \item La unión de los puntos de los clusters debe ser todo el conjunto de datos, es decir, $X = \bigcup c_i$. De nuevo, nuestra representación vectorial fuerza a que esta restricción se verifique
\end{itemize}

Por ejemplo, si tenemos 5 puntos y 3 clusters, una posible solución sería $\{3, 1, 2, 3, 0\}$. Y por otro lado, la solución $\{3, 1, 2, 3, 2\}$ no es válida pues el cluster $0$ está vacío.

Para cada solución podemos calcular algunas métricas necesarias para conocer el valor de $fitness$ de la solución que estamos representando. Para comenzar, por cada cluster podemos calcular el \textbf{centroide} del cluster:

\begin{displaymath}
    \vec{\mu_i} := \frac{1}{|c_i|} \sum_{x_i \in c_i} \vec{x_i}
\end{displaymath}

Definimos para cada cluster su \textbf{distancia media intra-cluster} como:

\begin{displaymath}
    \bar{c_i} := \frac{1}{|c_i|} \sum_{x_i \in c_i} || \vec{x_i} - \vec{\mu_i} ||_2
\end{displaymath}

Y con ello podemos definir la \textbf{desviación general de la partición} como:

\begin{displaymath}
    \bar{c} := \frac{1}{k} \sum_{i \in 1, \ldots k} \bar{c_i}
\end{displaymath}

Definimos $infeasibility$ como el número de restricciones, tanto del tipo \emph{Must Link} como del tipo \emph{Cannot Link}, que se violan.

Con ello, ya podemos definir el valor de $\lambda$ como $\lambda := \frac{D}{|R|}$ donde $|R|$ es el número total de restricciones y $D$ la distancia máxima entre dos puntos de $X$.


\subsubsection{Representación de la solución en código}

La solución se representa en la clase \lstinline{problem_datatypes::Solution}. El código de los campos del \lstinline{struct} desarrollado es:

\begin{lstlisting}[language=Rust, style=Boxed]
pub struct Solution<'a, 'b> {
    cluster_indexes: Vec<u32>,
    data_points: &'a DataPoints,
    constraints: &'b Constraints,
    number_of_clusters: i32,

    /// Representa el peso de infeasibility en el calculo de fitness
    /// Solo se calcula una vez al invocar a Solution::new
    lambda: f64,

    // Para cachear el valor de fitness pues es un calculo costoso de realizar
    // Como los datos del struct no cambian, podemos hacer el cacheo sin miedo
    // Usamos RefCell para tener un patron de mutabilidad interior
    fitness: RefCell<Option<f64>>,
}
\end{lstlisting}

Los campos del \lstinline{struct} representan:

\begin{itemize}
    \item \lstinline{cluster_indixes}: el vector solución que representa la asignación de puntos a clusters
    \item \lstinline{data_points}: referencia al conjunto de datos (sirve para calcular métricas como el $fitness$ de la solución que se representa)
    \item \lstinline{constraints}: referencia al conjunto de restricciones sobre los datos (sirve para calcular métricas como el $fitness$ de la solución que se representa)
    \item \lstinline{number_of_clusters}: número de clusters en los que se agrupan los datos (sirve para comprobar que una solución sea válida)
    \item \lstinline{lambda}: valor de $\lambda$, necesario para calcular el $fitness$
    \item \lstinline{fitness}: valor de $fitness$. Está incluida en un \lstinline{RefCell<Option<f64>>} para poder cachear su valor, puesto que los atributos de una instancia nunca cambian y el cálculo del valor $\lambda$ es muy costoso (implica calcular restricciones violadas y distancias entre puntos)
\end{itemize}

La comprobación de que no tenemos clusters sin puntos asignados se hace en el método \lstinline{Solution::is_valid}. La distancia media intracluster se calcula en \lstinline{Solution::intra_cluster_distance}. Mientras que la desviación general se calcula en \lstinline{Solution::global_cluster_mean_distance}. El valor de $infeasibility$ se calcula en \lstinline{Solution::infeasibility}. El cálculo de $\lambda$ se realiza en el \emph{constructor} del \lstinline{struct}.

\pagebreak

\section{Descripción de los algoritmos empleados}

\subsection{Búsqueda Local}

Usamos un pseudocódigo muy parecido a \lstinline{Python} pues es muy expresivo y facilita traducir partes de nuestro código real a pseudocódigo.

Método de exploración de entorno:

\begin{lstlisting}[language=Python, style=Boxed]
# Estrategia el primero mejor
# Devuelve el primer vecino que mejora la solucion actual
def get_neighbour():
    # Tomamos el generador de vecinos que se describe mas adelante
    neighbours_generator = generate_all_neighbours()

    # Mezclo los generadores de vecinos
    neighbours_generator.shuffle()

    # Exploro los vecinos hasta encontrar uno mejor que esta solucion
    for current_generator in neighbours_generator:
        current_solution = self.generate_solution_from(current_generator)

        if
            current_solution.is_valid() &&
            current_solution.fitness() < self.fitness():

            return current_solution

    # No hemos encontrado un vecino mejor
    return None;
}
\end{lstlisting}

Operador de generación de vecino:

\begin{lstlisting}[language=Python, style=Boxed]

# Struct que representa el generador de vecinos de
# forma eficiente
struct NeighbourGenerator:
    # El elemento que queremos mover de cluster
    element_index: i32,

    # El nuevo cluster al que asignamos el elemento
    new_cluster: u32,

# Funcion que genera todos los vecinos posibles de un elemento
# Los vecinos generados pueden ser no validos
def generate_all_neighbours(
    number_of_elements,
    number_of_clusters):
    neighbours = []

    for current_element in 0..number_of_elements:
        for current_cluster in 0..number_of_clusters:
            neighbours.append(NeighbourGenerator{
                current_element,
                current_cluster,
            });

    return neighbours;
\end{lstlisting}

Generación de soluciones aleatorias:

\begin{lstlisting}[language=Python, style=Boxed]
# Genera una solucucion inicial aleatoria como punto de partida de las busquedas
# Puede dejar clusters vacios, por lo que el caller de esta funcion tiene que
# comprobar la validez de la solucion aleatoria, y en caso de invalidez, volver
# a llamar a esta funcion (es muy poco probable que con muchos puntos dejemos
# un cluster vacio)
def generate_random_solution(data_points, constraints, number_of_clusters):
    # Vector con indices de clusters aleatorios, de tamaño el numero de puntos
    # que trabajamos
    random_cluster_indixes = [
        random(0, number_of_clusters)
        for _ in data_points.len()
    ]

    # En nuestro codigo, generamos el struct Solution a partir de los parametros
    # de entrada y random_cluster_indixes
    return solution_from(cluster_indexes)
}
\end{lstlisting}


\pagebreak

% TODO -- no se indica maximo pero por el apartado anterior pareciese razonable que fuesen solo dos paginas
% TODO -- creo que solo hay que poner pseudocodigo
\section{Descripción y Pseudocódigo de los algoritmos de comparación}

Como algoritmo de comparación estamos considerando una modificación del algoritmo clásico \emph{K-means} al que añadimos la capacidad de considerar las restricciones: \emph{copkmeans} o \emph{Constrained K-means}. Por tanto, estamos ante un algoritmo \emph{greedy}.

La idea general es:

\begin{enumerate}
    \item Partir de una solución inicial aleatoria, que vendrá dada por una asignación de centroides de clusters aleatorios
    \item Iterar sobre todos los datos en orden aleatorio, asignando a cada punto el mejor cluster en ese momento (siguiendo claramente un esquema greedy). Consideramos como mejor cluster el que menos restricciones violadas produzca, y en caso de empate, el cluster cuyo centroide sea más cercano al punto
    \item Una vez acabada la asignación de todos los puntos, calcular los centroides de los clusters con la asignación actual de los puntos
    \item Repetir el proceso desde 2 si los centroides han cambiado respecto de la anterior iteración
\end{enumerate}

A la hora de ejecutar el algoritmo, en algunos \emph{datasets} daba problemas, pues los centroides oscilaban infinitamente entre dos soluciones muy cercanas (debido entre otros factores a la configuración de los datos de entrada). Esta configuración de los datos también puede provocar que haya clusters que se queden sin puntos asignados, generando así una solución no válida. Por tanto, el algoritmo admite un parámetro de entrada para indicar si queremos que sea \emph{robusto} o no. En caso de que indiquemos que queremos que sea robusto se tendrán las siguientes diferencias:

\begin{itemize}
    \item Los centroides aleatorios no se tomarán como puntos aleatorios, sino como puntos del \emph{dataset} aleatorios, por lo que en una primera iteración no podrán quedar clusters vacíos, aunque si podrán quedar clusters vacíos en iteraciones posteriores. Con esto se buscas evitar el problema de los clusters vacíos
    \item Se tendrá un máximo de iteraciones. Este máximo lo hemos establecido como 50 iteraciones sobre el bucle principal. Teniendo en cuenta que cuando no cicla infinitamente, en menos de 10 iteraciones el algoritmo encuentra solución, consideramos que es un máximo mucho más que aceptable para asegurar que la solución devuelta sea la mejor (o la segunda mejor) que el \emph{greedy} puede calcular con esa semilla aleatoria. Con esto se busca evitar el problema del ciclado infinito
    \item Aún con estos cambios, en ciertas ocasiones no podemos evitar que dejemos un cluster vacío en iteraciones posteriores a la primera. Por tanto, también colocaremos un máximo de reinicios del algoritmo en la parte del código que llama al método de búsqueda.
\end{itemize}

Aún con estos cambios, en ciertas ocasiones no podemos evitar que dejemos un cluster vacío en iteraciones posteriores a la primera. Esto lo comentaremos más en profundidad en el análisis de los resultados

El pseudocódigo (en notación muy parecida a \lstinline{Python}) de nuestra implementación del algoritmo quedaría tal que:

\begin{lstlisting}[language=Python, style=Boxed]
# Generamos los centroides aleatorios. Dependiendo de si es robust o no
# consideramos puntos aleatorios en [0,1] x [0,1] o puntos del dataset
# de entrada aleatorios
current_centroids = generate_random_centroids()

# Solucion inicial aleatoria que se va a modificar en la primera iteracion
# Notar que no es valida porque deja todos los clusters menos uno vacíos
current_cluster_indixes = [0, 0, ..., 0]

# Para comprobar que los centroides cambien
centroids_have_changed = true


# Si robust == true, acotamos el numero maximo de iteraciones
max_iterations = 50
mut curr_iteration = 0

while centroids_have_changed == true and curr_iteration < max_iterations{

    # Realizamos una nueva asignacion de clusters. Recorremos los puntos
    # aleatoriamente y asignamos el cluster que menos restricciones viole
    # en esa iteracion. En caso de empate, se toma el cluster con centroide
    # mas cercano al punto
    new_cluster_indixes = assign_points_to_clusters()

    # Comprobamos que la nueva solucion calculada es correcta
    if valid_cluster_configuration(current_cluster_indixes) == false:
        # Esto hace que el el caller de la funcion copkmeans, se muestre un
        # mensaje por pantalla y se vuelva a realizar la busqueda, con lo que
        # partimos de unos centroides aleatorios nuevos. Como ya se ha comentado,
        # hay un maximo de reinicios en el caller para este metodo
        return None

    # Calculamos los nuevos centroides y comprobamos si han cambiado
    new_centroids = calculate_new_centroids(new_cluster_indixes)
    centroids_have_changed = centroids_are_different(
        current_centroids,
        new_centroids
    )

    # Cambiamos a la nueva asignacion de clusters y los nuevos centroides
    current_cluster_indixes = new_cluster_indixes
    current_centroids = new_centroids


    # En caso de que robust = true, acotamos el numero de iteraciones de forma
    # efectiva aumentando el contador. En otro caso, al no tocar el contador
    # no estamos teniendo en cuenta este parametro
    if robust == true:
        curr_iteration = curr_iteration + 1;

# Devolvemos la solucion en la estructura de datos correspondiente
return solution_from(current_cluster_indixes)
\end{lstlisting}

Desarrollamos el código de \lstinline{assign_points_to_clusters} por su importancia:

\begin{lstlisting}[language=Python, style=Boxed]
def assign_points_to_clusters():
    # Realizamos una nueva asignacion de clusters
    # -1 para saber que puntos todavia no han sido asignados a un cluster
    new_cluster_indixes= [-1, -1, ..., -1]

    # Recorremos aleatoriamente los puntos para irlos asignando a cada cluster
    point_indexes = (0..data_points.len())
    point_indexes.shuffle();

    for index in point_indexes:
        # Calculo el cluster al que asignamos el punto actual
        new_cluster_indixes[index] = select_best_cluster(
            current_cluster_indixes,
            current_centroids,
        )

    # Devolvemos los indices que representan la solucion
    return new_cluster_indixes
\end{lstlisting}

\pagebreak

\section{Explicación del procedimiento considerado para desarrollar la práctica}

Hemos desarrollado todo el código desde prácticamente cero usando el lenguaje de programación \lstinline{Rust}. Para el entorno de desarrollo, solo hace falta instalar \lstinline{Cargo} que permite compilar el proyecto, correr los ejecutables cómodamente, descargar las dependencias o correr los tests que se han desarrollado.

A continuación describimos el proceso de instalación y un pequeño manual de usuario para compilar y ejecutar el proyecto en \lstinline{Linux} (pues es el sistema operativo que nuestro profesor de prácticas, Daniel Molina, nos ha indicado que usa).

\subsection{Instalación del entorno}

Lo más sencillo y directo es instalar \lstinline{rustup} que se encargará de instalar todos los elementos necesarios para compilar y ejecutar la práctica. Para ello podemos ejecutar:

\begin{itemize}
    \item En cualquier distribución linux: \lstinline{curl --proto '=https' --tlsv1.2 -sSf https://sh.rustup.rs | sh} y seguir las instrucciones
    \item Por si acaso no tenemos actualizado el entorno: \lstinline{rustup update}
\end{itemize}

Una vez tengamos instalado \lstinline{rustup}, podemos ejecutar órdenes como \lstinline{cargo check}, \lstinline{rustc}, \ldots

\subsection{Compilación y ejecución del binario}

Además del binario aportado en la estructura de directios que se especifica en el guión, podemos generar fácilmente el binario con \lstinline{cargo} y ejecutarlo tal que:

\begin{itemize}
    \item Para compilar: \lstinline{cargo build --release} lo que nos generará un binario en \lstinline{./target/release/PracticasMetaheuristicas}
    \item Podemos usar ese binario para ejecutar el programa o podemos usar \lstinline{cargo run --release <parametros entrada>} para correr el programa de forma más cómoda
    \item Es muy importante la opción \lstinline{--release} porque de otra forma el binario no será apenas optimizado, lo que supondrá tiempos de ejecución mucho mayores
    \item Para correr los tests programados podemos hacer \lstinline{cargo test}
\end{itemize}

Si ejecutamos el binario sin parámetros, veremos por pantalla un mensaje indicando los parámetros que debemos pasar al programa. Estos parámetros son:

\begin{itemize}
    \item fichero de datos: el fichero donde guardamos las coordenadas de los puntos
    \item fichero de restricciones
    \item Semilla para la generación de números aleatorios
    \item Número de clusters en los que queremos clasificar los datos
    \item Tipo de búsqueda: para especificar el tipo de algoritmo que queremos ejecutar. Los posibles valores son:
    \begin{itemize}
        \item copkmeans: búsqueda greedy
        \item copkmeans\_robust: búsqueda greedy con los cambios ya indicados para que sea algo más robusto
        \item local\_search: búsqueda local
    \end{itemize}
\end{itemize}

\subsection{Compilación y ejecución usando el script}

El proceso de compilación y ejecución del programa, sobre los distintos conjuntos de datos y restricciones, usando las distintas semillas que más adelante se especifican, se puede lanzar de forma cómoda invocando el script \lstinline{./launch_all_programs}.

\pagebreak

\section{Experimentos y análisis realizados - Máximo sin especificar}

\subsection{Descripción de los casos del problema empleados}

\subsection{Resultados obtenidos según el formato especificado}

\subsection{Análisis de resultados}

\section{Bibliografía - Máximo sin especificar}


\end{document}
